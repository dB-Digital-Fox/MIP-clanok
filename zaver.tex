Na koniec si zhrňme, čo sme si v tejto práci opísali. Diabetes je choroba, ktorá v dnešnej dobe nemá riešenia a jedine ako ju zvládnuť je jednoducho prijať to. Avšak riešením by mohli byť moderné technológie a umelá inteligencia.
Vďaka výdobitkom modernej doby môžeme zlepšiť životný štýl tisícom a možno i miliónom ľudí, ktorý toutu zákernou chorobou trpia.

GIM je dobrý softvér, akšak má väčšie využitie pri laboratórnych výskumoch. Inteligentný systém na báze fuzzy logiky je možným riešením, avšak treba dbať na aspekt technológii a ich dostupnosť a cenu. Vynechal som aspekt umelej inteligencie zámerne, lebo tá v posledných rokoch moc nepokročila a percento presnosti sa zmenilo len o pár desatín, čo je úspech avšak pre korektnú implementáciu je treba dosiahnuť lepších výsledkov.

Nakoniec len toľko, že ako z jeden diabetikov, si uvedomujem vážnosť situácie a je to neustáli, každodenný boj.
Preto môžem povedať, využime výdobitky našej doby na pomoc druhým a nie na to, aby sme sa ich zbavovali.