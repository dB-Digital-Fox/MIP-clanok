Na koniec si zhrňme, čo sme si v tejto práci opísali. Diabetes je choroba, ktorá v dnešnej dobe nemá riešenia a jedine ako ju zvládnuť je jednoducho prijať to. Avšak riešením by mohli byť moderné technológie a umelá inteligencia.
Vďaka výdobytkom modernej doby môžeme zlepšiť životný štýl tisícom a možno i miliónom ľudí, ktorý touto zákernou chorobou trpia.

GIM je dobrý softvér, avšak má väčšie využitie pri laboratórnych výskumoch. Inteligentný systém na báze fuzzy logiky je možným riešením, avšak treba dbať na aspekt technológii a ich dostupnosť a cenu. Vynechal som aspekt umelej inteligencie zámerne, lebo tá v posledných rokoch moc nepokročila a percento presnosti sa zmenilo len o pár desatín, čo je úspech avšak pre korektnú a hlavne efektívnu implementáciu je treba dosiahnuť lepších výsledkov.

Nakoniec len toľko, že ako z jeden diabetikov, si uvedomujem vážnosť situácie a je to neustáli, každodenný boj.
Preto môžem povedať, využime výdobytky našej doby na pomoc druhým a nie na to, aby sme sa ich zbavovali.

\paragraph{Udržateľnosť a etika }
Udržateľnosť tohto systému musí byť maximálna. Raz po spustení sa predpokladá a očakáva zo strany pacientov, že systém bude aktualizovaný a využívaný. To znamená, že si treba dobre prepočítať, aká bude finančná záťaž, aby väčšina pacientov neodpadla kvôli nedostatku financií napríklad na senzory, dáta a iné veci s tým spojené.

Etika a teda aj bezpečnosť tohto systému bude veľmi vysoká, keďže ide o osobné údaje a životné funkcie, poprípade fyzické problémy pacienta. Tieto všetky dáta budú musieť byť bezpečnostne uchované, aby nedošlo k ich zneužitiu.
