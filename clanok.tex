% Metódy inžinierskej práce

\documentclass[10pt,slovak,a4paper]{article}

\usepackage[slovak]{babel}
%\usepackage[T1]{fontenc}
\usepackage[IL2]{fontenc} % lepšia sadzba písmena Ľ než v T1
\usepackage[utf8]{inputenc}
\usepackage{graphicx}
\usepackage{float}
\usepackage{caption}
\usepackage{subcaption}
\usepackage[utf8]{inputenc}
\usepackage{url} % príkaz \url na formátovanie URL
\usepackage{hyperref} % odkazy v texte budú aktívne (pri niektorých triedach dokumentov spôsobuje posun textu)

\usepackage{cite}
%\usepackage{times}

\pagestyle{headings}

\title{Softvér pre DIAbetikov\thanks{Semestrálny projekt v predmete Metódy inžinierskej práce, ak. rok 2021/22, vedenie: Vladimír Mlynarovič}} 

\author{Daniel Brilla\\[2pt]
	{\small Slovenská technická univerzita v Bratislave}\\
	{\small Fakulta informatiky a informačných technológií}\\
	{\small \texttt{xbrillad@stuba.sk}}
	}

\date{\small 12. december 2021}



\begin{document}

\maketitle

\begin{abstract}
Diabetes je v dnešnnom svete už bežnou chorobou, ktorou trpí čím ďalej, tým viac ľudí. Väčšina diabetikov používa bežne dostupné softvéry, ktoré avšak, podľa mojich názorov nedržia krok s dobou. Ponúkajú len dlhodobé sledovanie si hladiny cukru. V digitálnej dobe, plnej umelej inteligencie vieme, opäť podľa mojích názorov, sa dá posunúť vpred a zjednoduchšiť každodenný život. 

\paragraph{Kľúčové slová:} Diabetes, Fuzzy logika, Inteligentný softvér, GIM, Modifikované upozorňovacie skóre 
\end{abstract}



\section{Úvod}

V tejto semestralnej práci sa budem venovať problematike v softvérovom inžinierstve a ako zlepšiť život diabetikom po celom svete. S narastajúcim počtom diabetikov po celom svete prevencia už nepostačuje.
Je potrebné spraviť krok vpred a tím je zaoberanie sa otázkou, ako v dnešnom modernom svete zlepšiť každodenný život diabetikov? 
Táto problematika nemá také ľahké riešenia ako sa zdá. Teória je náročná a riešenia sú v stave testovania.

Na začiatok si povieme čo je diabetes a aké su jeho typy (dva základné). Následne sa pozrieme na niektoré softvérové riešenia ako GIM a alebo Inteligntné riešenie ktoré sú aktuálne testované.
Nakoniec si všetko zhrnieme a zhodnotíme z pohľadu diabetika, keďže od roku 2016 aj ja spadám do tejto kategórie. 






\section{Čo je diabetes} \label{diabetes}

Diabetes mellitus je metabolická disfunkcia charakterizovaná hyperglykémiou, ktorá je dôsledkom porúch sekrécie inzulínu z pankreasu, účinku inzulínu alebo spojením oboch porúch.\cite{2004}
Chronická hyperglykémia diabetu je spojená s dlhodobým poškodením, dysfunkciou a zlyhaním rôznych orgánov, najmä očí, obličiek, nervov, srdca a ciev.

\begin{figure}[h]
\includegraphics[scale=0.8]{pocet_diabetikov.png}
\caption{graf znázorňujúci počet diabetikov na 1000 obyvateľov podľa pohlavia od 1980 do 2017 \cite{2018}}
\end{figure}

Na vzniku cukrovky sa podieľa niekoľko faktorov. Tieto sa pohybujú od autoimunitnej deštrukcie beta buniek pankreasu s následným nedostatkom inzulínu po abnormality, ktoré vedú k rezistenci bunieki na pôsobenie inzulínu.
Základom abnormalít v metabolizme uhľohydrátov, tukov a bielkovín pri cukrovke je nedostatočné pôsobenie inzulínu na cieľové tkanivá. Nedostatočný účinok inzulínu je dôsledkom nedostatočnej sekrécie inzulínu a/alebo zníženej reakcie tkaniva na inzulín v jednom alebo viacerých bodoch komplexných dráh pôsobenia hormónov. Porušenie sekrécie inzulínu a defekty v pôsobení inzulínu často koexistujú u rovnakého pacienta a často nie je jasné, ktorá abnormalita, či už samotná, je primárnou príčinou hyperglykémie, prípadnej hypoglykémie.\cite{2004}

Medzi príznaky diabetesu patrí strata hmotnosti, alebo obezita, nadmerné močenie, smäd, hlad. Vážnejšími príznakmi sú napríklad zhoršenie zraku, problémy pri močené a iné.
K dlhodobým komplikáciám diabetu patrí retinopatia s potenciálnou stratou zraku; nefropatia vedúca k zlyhaniu obličiek; periférna neuropatia s rizikom vredov na nohách, amputácií a Charcotových kĺbov; a autonómna neuropatia spôsobujúca gastrointestinálne, genitourinárne a kardiovaskulárne symptómy a sexuálnu dysfunkciu. U pacientov s diabetom je zvýšený výskyt aterosklerotických kardiovaskulárnych, periférnych arteriálnych a cerebrovaskulárnych chorôb.\cite{2004}

Prevažná väčšina prípadov cukrovky spadá do dvoch širokých etiopatogenetických kategórií. Pri cukrovke 1. typu, je príčinou absolútny nedostatok sekrécie inzulínu. Jedinci so zvýšeným rizikom vzniku tohto typu diabetu môžu byť často identifikovaní sérologickými dôkazmi autoimunitného patologického procesu vyskytujúceho sa na ostrovčekoch pankreasu. Pri druhej, oveľa rozšírenejšej kategórii, cukrovke typu 2, je príčinou kombinácia odolnosti voči účinku inzulínu a neadekvátnej kompenzačnej sekrečnej reakcie na inzulín. V druhej kategórii môže stupeň hyperglykémie postačujúci na to, aby spôsobil patologické a funkčné zmeny v rôznych cieľových tkanivách, ale bez klinických symptómov, môže jedinec existovať dlhší čas, než sa zistí diabetes. Počas tohto asymptomatického obdobia je možné demonštrovať odchýlku v metabolizme uhľohydrátov meraním plazmatickej glukózy nalačno.\cite{2004}




\section{Aktuálny softvér pre diabetikov}

Keď si človek otvorí na svojom mobilnom zariadení, notebooku alebo počítači prehľad aplikácií pre diabetokv, všetko čo nájdeme sú primárne aplikácie určené pre sledovanie a analyzovanie už odmeraných hladín cukru v krvy (glykémie).
Veľká pomoc pre diabetikov, ktorý si so sebou nie vždy berú svoj diabetický denníkm, jeho papierovú verziu. Množstvo týchto aplikácii pracuje a komunikuje aj s RFID senzormi, ktoré, môže mať človek na sebe ako má inzulínovú pumpu alebo senzor na kontinuálne meranie krvy. 
Pomocou týchto softvérov si môže diabetik s pumpou, bez toho, aby vyberal veci naviac ( glukomer, pásiky, pichátko ) zmerať glykémiu a v prípade vysokej glykémie aj pripichnúť si inzulín.

Avšak, pri dnešnom modernom svete, ako si povieme neskôr\ref{int-sof}, umelá inteligencia (AI) v kombinácii s doterajšími poznatkami, by sa dalo predísť katastrofálnehším stavom a zmierniť vedľajšie komplikácie, ktoré s týmto ochorením idú ruka v ruke.
Nehovorím týmto, že tieto riešenia sú zlé. Podľa mňa sme za 100 rokov liečby diabetesu pokročili o míľové kroky vpred, avšak netreba sa zastaviť. 
\paragraph{Technológia a ľudia.}
Musíme napredovať a využívať vedomosti nie na zabíjanie ľudí a dokazovanie si moci, ale na pomoc druhým. 
Lebo pravá sila je ukrytá nie v tom, čo človek vie zobrať, ale čo vie druhému dať. A toto platí nielen pri diabetese ale aj pri ostatných, rovnako vážnych ochoreniach. 

Lebo toto je ľudský faktor. Vínimkou nie je nikto z nás. 


\section{GIM} \label{GIM}


GIM alebo Glucose-Insulin Model softvér nám dáva schopnosť simulovať chovanie sa jedinca a jeho sekréciu inzulínu.\cite{2007}

V poslednej dobe bol navrhnutý nový model simulácie jedla, ktorý umožnil meranie rôznych tokov, glukózy a inzulínu, vyskitujúcich sa počas jedla. V skutočnosti je systém, veľmi zložitý a iba dostupnosť tokov glukózy a inzulínu, ich plazmatických koncentrácií, nám umožní minimalizovať štruktúrne neistoty pri modelovaní rôznych procesov. Model pozostáva z 12 nelineárnych diferenciálnych rovníc, 18 algebraitických rovníc a 35 parametrov.\cite{2007}

Užívateľsky príjemný simulačný softvér tohto modelu by bol veľkou pomocou, najmä pre vyšetrovateľov bez konkrétnych odborných znalostí v oblasti modelovania.Softvér GIM, implementovaný v MATLAB verzii 7.0.1, ktorý umožňuje simulovať normálne aj patologické stavy, napr. Diabetes typu 2 a inzulín s otvorenou a uzavretou slučkou infúzie pri cukrovke 1. typu. Softvér sa nepokúša riešiť patofyziologické otázky.\cite{2007}

\subsection{MATLAB Version}

GIM softvér je založený na MATLAB-e. Ako je možné vidieť nižšie na uvedených obrázkoch(\ref{okna}), nie to to vôbec jednoduché na pochopenie. Tým pádom bežný pacient tenot softvér nevie využiť. Ak vás zujíma bližšie fungovanie tohto softvéru, prečítajte si o ňom v tejto správe zo sympózia, kde bol prvýkrát uvedený do používania \cite{2007}.

\begin{figure}[H]
\centering
 \begin{subfigure}[b]{0.475\linewidth}
   \includegraphics[width=\linewidth]{ob-3.PNG}
   \caption{Vystup}
 \end{subfigure}
 \begin{subfigure}[b]{0.475\linewidth}
   \includegraphics[width=\linewidth]{ob-4.PNG}
   \caption{vstup}
 \end{subfigure}
\caption{Dialogove okna v GIM \cite{2007}}
\label{okna}
\end{figure}

\paragraph{Grafické vyjadrenie informácií v informatike}
Viete v informatike, keď sa pripravuje nejaký softvér treba dbať aj na konečného užívateľa. Dobrým príkladom môže byť GIM. Jeho prostredie väčšine laikom povie akurát tak nič. Treba mať presne zadefinované, kto bude daný softvér používať. V prípade GIM simulátora sú to lekári a laboratórny pracovníci. Ak si vezmeme iný, voľne dostupný softvér pre diabetikov, je omnoho ústretovejší a prehľadnejší pre laikov, bežných ľudí.

\section{Inteligentný softvér} \label{int-soft}

V tejto kapitola sa chcem venovať kapitole z knihy Diabetes Technology and Therapeutics, ktorá ma zaujala kapitolou o inteligentnom diabetickom softvéry.

Pre diabetikov prvého typu je ťažké udržať si stálu alebo optimálnu hladinu cukru (/ref{diabetes}). Riešením by bol systém umelej inteligencie pozostávajúci z liečebných algoritmov kalibrovaných prostredníctvom veľkých súborov údajov špecifických pre pacienta.\cite{2000} Znie to zaujímavo, ale chybičku vidím v tom, ako sa ďalj píše v knihe, že je to potrebné spsraviť po každej zmene, či už denného režimu, inzulínu alebo športových aktivít. Samotné nastavenie systému nie je je v prototype jednoduché a pre množstvo užívateľov neprípustné. 

Softvérový prototyp založený na neurónovej sieti, fuzzy logike a konceptoch expertného systému bol vyvynutý a hodnotený na určenie uskutočniteľnosti a účinnosti predikčného modelu špecifického pre pacienta. Priemerná absolútna percentuálne chyba medzi skutočnými a predpokladanými hodnotami glykémie (hladiny cukru v krvy) zo vstupov denného inzulínu, jedla a informácií o cvičení u testovacích subjektov s Diabetes Melitus 1 bola 10.5 precenta.\cite{2000}
Zdá sa to ako celkom veľká odchylka, čo aj je, avšak je to stále dosť presné nato, aby to dokázalo zabrániť životu nebezpečným situáciám. 

\subsection{Fuzzy Logika v zdravotníctve}

Typickému zhoršeniu stavu u chorých ľudí predchádzajú rôzne fyzioôogické zmeny ako pulz alebo krvný tlak. Modifikované skoré upozorňovacie skóre je systém, ktotrý bol vyvynutý na assistenciu nemocničnému personálu pri meraní týchto zmien a pri identifikácii pacientov, ktorý potrebujú naliehavú lekársku pomoc aby sa predišlo katastrofálnym stavom. \cite{2019} Systém je aktuálne implementovaný a testovaný v Rashid Center for Diabaetes and Research v UAE.

Primárne požiadavky systému je diaľkový zber vitálnych funkcií pacienta, ktoré sa merajú pomocou snímačov na báze RFID a hodnotenie zdravotného stavu pacienta pomocou algoritmov založených na fuzzy logike.  Tieto dáta sú následne uchované v elektronických lekárskych záznamoch (EMR) a upozorniť zdravotnícky personál o pacientovom statuse a či potrebuje urgentnú staroslivosť alebo nie. \cite{2019} Následný diagram nám ukazuje schému navrhovaného systému.

\begin{figure}[h]
\includegraphics[scale=0.5]{scheme-fuzzy.png}
\caption{Schéma navrhovaného systému \cite{2019}}
\end{figure}

RFID senzory môžu komunikovať bezdrôtovo s mobilným zariadením a následne pomocou mobilnej siete prenášať infomácie do centrálneho monitorovaacieho systému - počítača. Tento systém umožňuje implementáciu viacerych užívateľov ako aj počet oblastí monitorovania. \cite{2019} 

Systém pozostáva z rôznych softvérových modulov. Ide o tieto moduly: programovateľné rozhranie pre jednotku zberu dát – aplikácia (DAQ-API), fuzzy logický engine (FLE), databázový manažér (DM), grafické používateľské rozhranie (GUI) a webová aplikácia (Web- Aplikácia). Podrobnosti a funkcie každého softvérového modulu sú zhrnuté nižšie. \cite{2019}

Modul DAQ-API umožňuje interakciu s čítačkou RFID s cieľom zhromažďovať vitálne funkcie pacienta v reálnom čase a zobrazovať ich na obrazovke API. Je to front-end monitorovacie a prevádzkové rozhranie pre používateľov systému.\cite{2019}

DM, GUI a Web-App boli vyvinuté ako echo moduly. Používajú sa na profilovanie používateľov, ukladanie životných funkcií a umožňujú im interakciu so systémom cez web. Podrobný popis týchto echo modulov možno nájsť v štúdii Al-Damour\cite{2013}.\cite{2019}

Modul fuzzy logiky má tri postupné procesy: fuzzyfikácia, systém založený na pravidlách a proces defuzzyfikácie.\cite{2019}

 \begin{figure}[h]
\includegraphics[scale=0.5]{rule-based-engine.png}
\caption{Schéma fuzzy procesu podľa článku\cite{2019}}
\label{fuzzy}
\end{figure}

V schéme \ref{fuzzy} vstupujú dátat z RFID senzorov a výsledkom je jedno číslo (status) stavu pacienta. Paralelný charakter pravidiel je jedným z dôležitejších aspektov systémov fuzzy logiky. Namiesto ostrého prepínania medzi režimami na základe bodov prerušenia, logika plynulo prúdi z oblastí, v ktorých správaniu systému dominuje jedno alebo druhé pravidlo. Systém je implementovaný pomocou MATLAB Fuzzy Logic Toolbox. \cite{2019}

\subsection{zhrnutie}
Inteligntné systému sú riešením avšak s technológiou prichádza aj daň. Systém sa zdá byť na povrchu jednoduchý avšak pre čo najrýchlejšie spracovanie údajov je potreba celkom veľké množstvo výpočtovej techniky a to nerozprávam len o počítačoch na výpočet. Senzory nevydržia dlho. Abbott FreeStyle Libre, čipy, senzory pre kontinuálne sledovanie hladiny cukru, tiež nedržia večne. A ďalšou požiadavkou bolo pripojenie na mobilnú sieť aby sa dáta dali nahrať na webstránku. Toto je síce menší problím, keďže dáta sa dajú ukladať lokálne a následne naraz odoslať do systému pre spracovanie. Je tam pár háčikov ale blížime sa.


\section{Záver} \label{zaver}
Na koniec si zhrňme, čo sme si v tejto práci opísali. Diabetes je choroba, ktorá v dnešnej dobe nemá riešenia a jedine ako ju zvládnuť je jednoducho prijať to. Avšak riešením by mohli byť moderné technológie a umelá inteligencia.
Vďaka výdobitkom modernej doby môžeme zlepšiť životný štýl tisícom a možno i miliónom ľudí, ktorý toutu zákernou chorobou trpia.

GIM je dobrý softvér, akšak má väčšie využitie pri laboratórnych výskumoch. Inteligentný systém na báze fuzzy logiky je možným riešením, avšak treba dbať na aspekt technológii a ich dostupnosť a cenu. Vynechal som aspekt umelej inteligencie zámerne, lebo tá v posledných rokoch moc nepokročila a percento presnosti sa zmenilo len o pár desatín, čo je úspech avšak pre korektnú implementáciu je treba dosiahnuť lepších výsledkov.

Nakoniec len toľko, že ako z jeden diabetikov, si uvedomujem vážnosť situácie a je to neustáli, každodenný boj.
Preto môžem povedať, využime výdobitky našej doby na pomoc druhým a nie na to, aby sme sa ich zbavovali. % prípadne iný variant názvu



%\acknowledgement{Ak niekomu chcete poďakovať\ldots}


% týmto sa generuje zoznam literatúry z obsahu súboru literatura.bib podľa toho, na čo sa v článku odkazujete
\newpage
	\bibliography{literatura1.bib}
	\bibliographystyle{ieeetr} % prípadne alpha, abbrv alebo hociktorý iný
\end{document}
