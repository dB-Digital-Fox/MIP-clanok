Diabetes mellitus je metabolická disfunkcia charakterizovaná hyperglykémiou, ktorá je dôsledkom porúch sekrécie inzulínu z pankreasu, účinku inzulínu alebo spojením oboch porúch.\cite{2004}
Chronická hyperglykémia diabetu je spojená s dlhodobým poškodením, dysfunkciou a zlyhaním rôznych orgánov, najmä očí, obličiek, nervov, srdca a ciev.

Na vzniku cukrovky sa podieľa niekoľko faktorov. Tieto sa pohybujú od autoimunitnej deštrukcie beta buniek pankreasu s následným nedostatkom inzulínu po abnormality, ktoré vedú k rezistenci bunieki na pôsobenie inzulínu.
Porušenie sekrécie inzulínu a defekty v pôsobení inzulínu často koexistujú u rovnakého pacienta a často nie je jasné, ktorá abnormalita, či už samotná, je primárnou príčinou hyperglykémie, prípadnej hypoglykémie.\cite{2004}

Medzi príznaky diabetesu patrí strata hmotnosti, alebo obezita, nadmerné močenie, smäd, hlad. Vážnejšími príznakmi sú napríklad zhoršenie zraku, problémy pri močené a iné.
K dlhodobým komplikáciám diabetu patrí retinopatia s potenciálnou stratou zraku, nefropatia vedúca k zlyhaniu obličiek, periférna neuropatia s rizikom vredov na nohách. U pacientov s diabetom je zvýšený výskyt aterosklerotických kardiovaskulárnych, periférnych arteriálnych a cerebrovaskulárnych chorôb.\cite{2004}

Prevažná väčšina prípadov cukrovky spadá do dvoch širokých etiopatogenetických kategórií. Pri cukrovke 1. typu, je príčinou absolútny nedostatok sekrécie inzulínu. Pri druhej, oveľa rozšírenejšej kategórii, cukrovke typu 2, je príčinou kombinácia odolnosti voči účinku inzulínu a neadekvátnej kompenzačnej sekrečnej reakcie na inzulín.\cite{2004}


