V tejto semestralnej práci sa budem venovať problematike v softvérovom inžinierstve a ako zlepšiť život diabetikom po celom svete. S narastajúcim počtom diabetikov po celom svete prevencia už nepostačuje.
Je potrebné spraviť krok vpred a tím je zaoberanie sa otázkou, ako v dnešnom modernom svete zlepšiť každodenný život diabetikov? 
Táto problematika nemá také ľahké riešenia ako sa zdá. Teória je náročná a riešenia sú v stave testovania.

Na začiatok si povieme čo je diabetes a aké su jeho typy (dva základné). Následne sa pozrieme na niektoré softvérové riešenia ako GIM a alebo Inteligntné riešenie ktoré sú aktuálne testované.
Nakoniec si všetko zhrnieme a zhodnotíme z pohľadu diabetika, keďže od roku 2016 aj ja spadám do tejto kategórie. 




