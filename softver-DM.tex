Keď si človek otvorí na svojom mobilnom zariadení, notebooku alebo počítači prehľad aplikácií pre diabetokv, všetko čo nájdeme sú primárne aplikácie určené pre sledovanie a analyzovanie už odmeraných hladín cukru v krvy (glykémie).
Veľká pomoc pre diabetikov, ktorý si so sebou nie vždy berú svoj diabetický denníkm, jeho papierovú verziu. Množstvo týchto aplikácii pracuje a komunikuje aj s RFID senzormi, ktoré, môže mať človek na sebe ako má inzulínovú pumpu alebo senzor na kontinuálne meranie krvy. 
Pomocou týchto softvérov si môže diabetik s pumpou, bez toho, aby vyberal veci naviac ( glukomer, pásiky, pichátko ) zmerať glykémiu a v prípade vysokej glykémie aj pripichnúť si inzulín.

Avšak, pri dnešnom modernom svete, ako si povieme neskôr\ref{int-sof}, umelá inteligencia (AI) v kombinácii s doterajšími poznatkami, by sa dalo predísť katastrofálnehším stavom a zmierniť vedľajšie komplikácie, ktoré s týmto ochorením idú ruka v ruke.
Nehovorím týmto, že tieto riešenia sú zlé. Podľa mňa sme za 100 rokov liečby diabetesu pokročili o míľové kroky vpred, avšak netreba sa zastaviť. 
\paragraph{Technológia a ľudia.}
Musíme napredovať a využívať vedomosti nie na zabíjanie ľudí a dokazovanie si moci, ale na pomoc druhým. 
Lebo pravá sila je ukrytá nie v tom, čo človek vie zobrať, ale čo vie druhému dať. A toto platí nielen pri diabetese ale aj pri ostatných, rovnako vážnych ochoreniach. 

Lebo toto je ľudský faktor. Vínimkou nie je nikto z nás. 